\let\proof\relax
\let\endproof\relax
\usepackage{amsmath,amsthm,amssymb,amsfonts,color,bm}
\usepackage[pdftex]{graphicx}
%\usepackage{cite}
\usepackage{mathtools}
\usepackage{physics}
\let\norm\relax
\DeclarePairedDelimiter{\norm}{\lVert}{\rVert}
\newcommand{\pushright}[1]{\ifmeasuring@#1\else\omit\hfill$\displaystyle#1$\fi\ignorespaces}
\newcommand{\pushleft}[1]{\ifmeasuring@#1\else\omit$\displaystyle#1$\hfill\fi\ignorespaces}
\newcommand{\cu}[1]{\mathbf{#1}}
\newcommand{\tash}[2]{\frac{\partial #1}{\partial #2}}
\newcommand{\tashh}[2]{\frac{\partial^2 #1}{\partial {#2}^2}}
\newcommand{\innerp}[2]{\langle{#1}, {#2}\rangle}

\newcommand{\bracket}[2]{\left\{\begin{matrix}
		#1\\
		#2
	\end{matrix}\right.}
\newcommand{\easytwo}[2]{\begin{bmatrix}
		#1\\ 
		#2
\end{bmatrix}}
\newcommand{\easyfour}[4]{\begin{bmatrix}
		#1& #2\\ 
		#3& #4
\end{bmatrix}}
\newcommand{\easynine}[9]{\begin{bmatrix}
		#1& #2& #3\\ 
		#4& #5& #6\\
		#7& #8& #9
\end{bmatrix}}
\newcommand{\fastmatrix}[1]{\begin{pmatrix}#1\end{pmatrix}}

%%% Functions

\DeclareMathOperator*{\argmin}{arg\,min}  % Argmin
\DeclareMathOperator*{\argmax}{arg\,max}  % Argmax

%\DeclareMathOperator\erf{erf} % Standard error function

%%% Probability

\newcommand{\expec}{\mathbb{E}}           % Expectation
\newcommand{\cov}{\operatorname{Cov}}     % Covariance
\newcommand{\varr}{\operatorname{Var}}     % Variance
\newcommand{\gauss}{\mathcal{N}}          % Gaussian distribution

%%% Number sets

\newcommand{\R}{\mathbb{R}} % Set of real numbers
\newcommand{\Q}{\mathbb{Q}} % Set of rational numbers
\newcommand{\N}{\mathbb{N}} % Set of natural numbers
\newcommand{\Z}{\mathbb{Z}} % Set of integers

%%% Standard constants and objects
\DeclareMathOperator{\neper}{e} % Exponential constant

%% Linear algebra operations
\newcommand{\transpose}{\mathsf{T}}         % Transpose of a matrix
%\newcommand{\trace}{\operatorname{tr}}      % Matrix trace
\newcommand{\adj}{\operatorname{adj}}       % Matrix adjugate
\newcommand{\diag}{\operatorname{diag}}     % Diagonal matrix
%\newcommand{\rank}{\operatorname{rank}}     % rank of a matrix
\newcommand{\Hess}{\operatorname{Hess}}     % Hessian matrix
\newcommand{\lspan}{\operatorname{span}}    % linear span
\newcommand{\mineig}{\lambda_{\text{min}}}  % Minimal eigenvalue
\newcommand{\maxeig}{\lambda_{\text{max}}}  % Maximal eigenvalue

%% Complement

% \newtheorem{theorem}{Theorem}